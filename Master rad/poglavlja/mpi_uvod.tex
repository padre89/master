\chapter{MPI-2 стандард}

\section{MPI}
Као и \textit{Lustre} фајл систем, и \gls{MPI} има за циљ да побољша паралелизам. Најоптималнији учинак се постиже истовременом применом MPI стандарда и \textit{Lustre} фајл система.
\gls{MPI} (Message-Passing Interface) је стандард за писање паралелних програма. MPI је развијан у две фазе, од стране произвођача паралелних рачунара, писаца библиотека и програмера апликација. Прва фаза је била 1993-1994 и резултат ове фазе је прва верзија MPI стандарда, названа  MPI-1. 
Један број важних тема у паралелном рачунарству је намерно изостављен из MPI-1, како би се убрзао излазак нове верзије ове библиотеке. MPI форум се састао 1995. да би се размотриле ове теме као и извршавање мањих исправки и појашњења која су се појавила у MPI-1. Верзија стандарда MPI-2 изашла је у лето 1997.

Почевши са радионицама 1992. године, MPI форум формално је организован 1992. године. MPI стандард је успео да се развије захваљујући привлачењу пажње широког спектра заједнице паралелног рачунарства. На окупљањима, произвођачи паралелних компјутера су слали најбоље техничко особље. MPI форум се одржавао сваких шест недеља, почевши од јануара 1993, а прва верзија MPI је изашла већ у лето 1994.

Прва акција форума је била да исправи грешке и разјасни низ питања која су изазивала неспоразуме у оригиналном документу из јула 1994, који је означен као MPI-1.0. Све ове измене су заокружене у целину и у мају 1995. изашао је MPI-1.1. Исправке и појашњења су се наставила и следеће две године. Резултат тог рада је MPI-2 документ, који као поглавље садржи и верзију MPI-1.2. 
У наредним поглављима описан је стандардни метод за покретање MPI програма, а затим паралелне улазно/излазне операције у MPI-2. Значајно је напоменути да MPI 2 омогућава да процес директно приступа подацима другог процеса.
Да би се увидела разлика између \textit{Lustre} и NFS фајл система, покретани су програми који мере брзине улазно/излазних операција. Поређењем резултата уочава се да је \textit{Lustre} фајл систем погоднији за паралелне програме.
\newpage

