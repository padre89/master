\chapter{ \textit{Увод} }
%\section{Увод}

Овај мастер рад као примарни циљ поставља проучавање могућности \textit{Lustre} фајл система у унапређењу рада кластера високих перформанси, као и користи од паралелних улазно излазних операција дефинисаних MPI-2 стандардом.
\textit{Lustre} паралелни фајл систем има предност над осталим (секвенцијалним) системима услед могућности обављања улазно-излазних операција у паралелном маниру. Са друге стране, MPI-2 представља индустријски стандард за развој паралелних програма, а његове имплементације подржавају паралелно читање и писање на уређаје масовне меморије. У циљу сагледавања начина да се побољшају укупне перформансе локалног кластера \textit{Мedflow}, упоређују се брзине читања и писања података у фајлове који се налазе како на \textit{Lustre}, тако и на NFS  фајл систему.

Кластер високих перформанси (\textit{High Performance Computing Cluster}) представља скуп рачунара умрежених коришћењем локалне мрежне инфраструктуре, помоћу које међусобно комуницирају. Коришћење специфичне програмске подршке даје висок степен интеграције рачунара, омогућава њихов координирани заједнички рад и претвара их ефективно у јединствен вишепроцесорски систем који користи дистрибуирану меморију.

За истраживачке радове који захтевају обимне математичке прорачуне данас се углавном користе паралелни програми који на кластеру високих перформанси рашчлањују проблем на више рачунарских процесора. Тиме се време добијања резултата знатно смањује у односу на време потребно за добијање резултата на једном процесору, понекад за више редова величине. Најефективнији начин је употреба поменутог MPI стандарда за развој паралелних програма. На Универзитету у Крагујевцу постоји неколико HPC кластера, али ни један од њих не поседује паралелни фајл систем. Та чињеница у појединим случајевима употребе може да доведе до озбиљног пада перформанси. На срећу, први кластер са паралелним фајл системом постављен је недавно у истраживачко развојном центру за биоинжееринг у Крагујевцу. Након иницијалне инсталације HPC кластера, дошло се на идеју да се истраже реалне могућности и употребна вредност новоинсталираног \textit{Lustre} система, што и представља основну мотивацију за израду овог мастер рада.
 
Рад је подељен у пет поглавља, а у наставку је дат кратак садржај. У првом делу рада се описују компоненте \textit{Lustre} фајл система, као и тестна инсталација система на скупу виртуалних машина. Затим се објашњава специфично дељење фајлова у систему и улога појединих сервиса. Управо је дељење фајлова и паралелни приступ оно што разликује \textit{Lustre} фајл систем од осталих. Најподобније је описана инсталација система услед значајне разлике у односу на инсталације класичних фајл система који су подржани директно од стране \textit{Linux} кернела. У поглављу 2.7 су описани алати помоћу којих се  најоптималније врше подешавања \textit{Lustre} фајл система. 
У другом делу рада се описује MPI-2 стандард и његове могућности везане за паралелне улазно/излазне операције. Испитују се начини преко којих је могуће вршити паралелне улазно/излазне операције и њихове особености.
У трећем делу рада су дата тестирања брзина извршавања операција фајл система помоћу програма \textit{Iozone}, \textit{dd} и \textit{Game of Life}. Под претпоставком да постоји разлика у брзини улазно/излазних операција ових фајл система, анализирају се добијени резултати тестирања и на основу њих се утврђује који је погоднији за који домен употребе.

\newpage
