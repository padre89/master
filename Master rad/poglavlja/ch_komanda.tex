\section{Кориснички алати за подешавање система}
\subsection{Алати за откривање грешака} 
  \textit{Lustre}  је сложен систем и може се наићи на разнолике проблеме при коришћењу. Треба имати алат за откривање грешака при руци, који може помоћи у схватању проблема и разлога његовог настанка. Разноврсни алати за дијагностику и анализе су доступни приликом отклањања проблема са   \textit{Lustre}  софтвером. Неки од њих се налазе у \textit{Linux} дистрибуцијама, док су
други  развијени и доступни су у оквиру    \textit{Lustre} пројекта. Следећи  \textit{in-kernel} механизми за отклањање грешака  су уграђени у   \textit{Lustre}  софтвер:

\begin{itemize}
\item  \textit{Debug logs}

\item  \textit{Debug} позадински сервис

\item  \textit{/proc/sys/lnet/debug}
\end{itemize}
Следећи програми су такође укључени у   \textit{Lustre} :
\begin{itemize}
\item \textit{lctl}

\item   \textit{Lustre subsystem asserts}

\item \textit{lfs}
\end{itemize}
Поред њих, ту су и општи алати који су уграђени у стандардне \textit{Linux} дистрибуције:
\begin{itemize}
\item \textit{strace}

\item \textit{/var/log/messages}

\item \textit{Crash dumps}

\item debugfs
\end{itemize}
Следећи \textit{logging} алати могу се користити за скупљање информација приликом откривања   \textit{Lustre}  \textit{kernel} проблема.
\begin{itemize}
\item \textit{kdump}

\item \textit{netconsole}

\item \textit{netdump}
\end{itemize}

За отклањање грешака   \textit{Lustre}  система у развојном окружењу користи се \textit{leak\_finder.pl}, док се за отклањање грешака и анализу користе
\begin{itemize}
\item \textit{kgdb}
\item \textit{crash}.
\end{itemize}

% МИЛОШ: Овде опет фали неки везивни текст. 
Кориснички програми:
\begin{itemize}
\item \textit{mkfs.lustre}
\item \textit{tunefs.lustre}
\item \textit{lctl}
\item \textit{mount.lustre}
\item \textit{lustre\_rsync}
\item Додатни кориснички програми за подешавање конфигурације
\end{itemize}

\subsection{mkfs.lustre}
\textit{mkfs.lustre} je кориснички програм који форматира диск за   \textit{Lustre} фајл систем.

Синтакса ове команде је:
\begin{verbatim}
mkfs.lustre <target_type> [options] device
\end{verbatim}
где је \textit{<target\_type>} један од следећих типова:

\begin{itemize}
\item \textbf{--ost}
\textit{Object Storage Target (OST)}
\item \textbf{--mdt}
\textit{ Metadata Storage Target (MDT)}
\item \textbf{--network=net }
врста мреже којом су повезани OST и MDT
\item \textbf{--mgs }
\textit{ Configuration Management Service (MGS)} може бити комбинован са  --mdt
\end{itemize}
После форматирања диска, диск може бити коришћен од стране   \textit{Lustre} сервиса.
Када је диск креиран, истом командом се  могу подесити параметри  одговарајућег диска.
Параметри се додају као \textit{--param} опција \textit{mkfs.lustre} команде.

Параметри \texttt{mkfs.lustre} команде су:

\begin{itemize}
\item \textbf{\textit{--backfstype=fstype }}
Поставља посебан формат за резервне копије фајл система (као што су \textit{ext3}, \textit{ldiskfs}).
\item \textbf{\textit{ --comment }}
Поставља кориснички коментар.
\item \textbf{\textit{--device-size=KB }}
Поставља величину уређаја.
\item \textbf{\textit{--dryrun }}
Штампа шта ће бити учињено. Не утиче на операције са диском.
\item \textbf{\textit{--failnode=nid }}
Поставља Ид партнера за отказе.
\item \textbf{\textit{--fsname=filesystem\_name }}
Поставља назив   \textit{Lustre} фајл система којег ће бити део. Подразумевани је \zn lustre”. Назив може имати највише 8 карактера.
\item \textbf{\textit{--index=index }}
Поставља OST или MDT индекс.
\item \textbf{\textit{--mkfsoptions=opts }}
Поставља опције за резервне копије.  Опције могу бити постављене овде су ext3.
\item \textbf{\textit{--mountfsoptions=opts} }
Поставља опције које се користе приликом монтирања диска.
\item \textbf{\textit{--mgsnode=nid }}
Постављају NID за MGS чвор.
\item \textbf{\textit{--param sys.timeout=40 }}
Поставља тајм-аут система.
\item \textbf{ \textit{--param lov.stripesize=2M }}
Поставља stripe количину меморије.
\item \textbf{\textit{--param lov.stripecount=2 }}
Поставља број stripe делова.
\item \textbf{\textit{--param failover.mode=failout }}
Враћа грешке диска, уместо да чека на опоравак.
\item \textbf{\textit{--reformat }}
Реформатира диск.

\end{itemize}

\subsection{\textit{tunefs.lustre}}

Кориснички програм који мења конфигурацију на   \textit{Lustre} диску је \textit{tunefs.lustre}.
Синтакса ове команде је:
\begin{verbatim}
tunefs.lustre [options] device
\end{verbatim}

Овом командом се може инсталирати најновија верзија   \textit{Lustre} система, с тим што се приликом извршавања ове команде не брише садржај диска. Измене после извршавања команде се могу уочити након следећег подизања диска.

\textit{tunefs.lustre} поставља параметре додавањем нових или мењањем старих. За брисање старих параметара и додавање нових потребно је покренути:
\begin{verbatim}
 tunefs.lustre --erase-params --param=<new parameters>
 \end{verbatim}
Параметри \textit{tunefs.lustre} команде су:
\begin{itemize}

\item \textbf{\textit{--comment=comment }}
Поставља кориснички коментар.
\item \textbf{\textit{--dryrun}} 
Штампа шта ће бити учињено. Не утиче на операције са диском.
\item \textbf{\textit{--erase-params} }
Брише све претходне параметре.
\item \textbf{\textit{--failnode=nid} }
Поставља Ид партнера за отказе
\item \textbf{\textit{--fsname=filesystem\_name }}
Поставља назив   \textit{Lustre}  фајл система којег ће бити део. Подразумевани је \zn \textit{lustre}”.
\item \textbf{\textit{--index=index }}
Поставља OST или MDT индекс.

\item \textbf{\textit{--mkfsoptions=opts }}
Поставља опције за резервне копије. Опције које могу бити постављене овде су ext3.
\item \textbf{\textit{--network=net }}
Поставља мрежу преко које су повезани OST/MDT чворови.
\item \textbf{\textit{--mgs }}
Додаје сервис за управљање конфигурацијом.
\item \textbf{\textit{--msgnode=nid,... }}
Поставља  NID(s) за MGS чвор
\item \textbf{\textit{--nomgs} }
Уклања сервис за управљање конфигурацијом.
\item \textbf{\textit{--quiet }}
Штампа основних информације.

\item \textbf{\textit{--verbose }}

Штампа више информације.
\item \textbf{\textit{--writeconf }}
Брише све конфигурационе дневнике за фајл систем на коме је MDT, а затим их регенерише. Приликом извршавања ове команде може доћи до престанка рада сервера и клијената. У општем случају, ова команда треба да буде извршена само на MDT.
\end{itemize}

\subsubsection{\textit{lctl}}
Кориснички програм који се користи за \textit{root} контролу и конфигурацију назива се \textit{lctl}. Са \textit{lctl} може се директно контролисати   \textit{Lustre} , омогућавајући различите конфигурације.
\begin{verbatim}
lctl
lctl --device <OST device number> <command [args]>
 \end{verbatim}
Најчешће \textit{lctl} команде су:
\begin{itemize}

\item \textbf{\textit{dl} }
Излистава   \textit{Lustre} уређаје по називу и броју. Команда штампа и UUID уређаја. На серверу, UUID је различит за све уређаје, док код клијента UUID је исти за све уређаје који су  део исте тачке подизања фајл система.
\item \textbf{\textit{device} }
Селектује задати OBD уређај. Све команде после селектовања уређаја зависиће од њега

\item \textbf{\textit{network <up/down>} }
Укључује или искључује LNET. Овом командом је могуће и одређивање типа мреже за остале \textit{lctl} LNET команде.
\item \textbf{\textit{list\_nids} }
Штампа све NIDs на локалном чвору. 
\item \textbf{\textit{ping {nid}} }
Проверава LNET умреженост користећи LNET ping.
\item \textbf{\textit{help} }
Штампа листу свих могућих  lctl команди. 
\item \textbf{\textit{conn\_list }}
Штампа све умрежене чворове.
 \item \textbf{\textit{route\_list }}
 Штампа комплетну табелу рутирања.
\end{itemize}

\subsection{\textit{mount.lustre}}

Кориснички програм који покреће   \textit{Lustre} клијента или други сервис је 
\textit{mount.lustre}.
\begin{verbatim}
mount -t lustre [-o options] directory
 \end{verbatim}

Овај програм не треба позивати директно. Ово је помоћни програм програма \textit{mount}. Командом \textit{umount} стопирамо   \textit{Lustre} сервис. Постоје 2 облика подизања фајл система:

\begin{itemize}

\item \textbf{\textit{<mgsspec>:/<fsname>} }
Подиже   \textit{Lustre} фајл систем контактирајући МGS \textit{<mgsspec>} на директоријуму \textit{<directory>}. Подигнути фајл систем постоји у  \textit{fstab-у} и он је доступан као и остали локални фајл системи.

\item \textbf{\textit{<disk\_device> }}
Покреће сервис дефинисан \textit{mkfs.lustre} командом на физичком диску \textit{<disk\_device>}. Подигнути фајл систем је доступан само за \textit{df} операције. Постоји у \textit{fstab-у}, показујући да је он у употреби.
\end{itemize}

\subsection{\textit{ lustre\_rsyncl}}
\textit{lustre\_rsync} програм синхронизује   \textit{Lustre} фајл систем. Дизајниран је тако да синхронизује систем са другим фајл системом. Тај фајл систем може бити било који други фајл систем. Операција синхронизације је ефикасна и не захтева претрагу измена, већ користи MDT логове измена да идентификује измене у систему. 

Опције овог програма су:

\begin{itemize}
\item \textbf{\textit{--source=<src> }}
Путања до \textit{root-а }  \textit{Lustre} система  који ће бити синхронизован.
\item \textbf{\textit{--target=<tgt> }}
Путања до фајл система где се синхронизује \textit{root}   \textit{Lustre} система.
\item \textbf{\textit{--mdt=<mdt> }}
Извршити синхронизацију MDT-а.
\item \textbf{\textit{--user=<user id>}}
Ид корисника логова измена. Да би се вршила синхронизација, корисник мора бити регистрован.
\item \textbf{\textit{--statuslog=<log> }}
Лог фајл у коме ће се чувати статус синхронизације.
\item \textbf{\textit{--dry-run }}
Штампа излаз \textit{lustre\_rsyncl} команде (\textit{cp}, \textit{mkdir}, итд.).
\item \textbf{\textit{--abort-on-err }}
Уколико дође до грешке прекида операцију синнхронизације.
\end{itemize}


\subsection{Додатни кориснички програми за подешавање конфигурације}

\subsubsection{\textit{lustre\_rmmod.sh}}
Брише све   \textit{Lustre}  и LNET модуле.

\subsubsection{e2scan}
Кориснички програм који претражује фајл систем у циљу проналаска последње измењених фајлова назива се \textit{e2scan}. \textit{e2scan} користи \textit{libext2fs} да пронађе фајлове са новијим \textit{mtime} и \textit{ctime} од задатог. Користећи овај програм, веома се ефикасно може добити листа фајлова који су измењени.

\begin{verbatim}
e2scan [options] [-f file] block_device
 \end{verbatim}
e2scan скенира све чворове на уређају, проналази измењене фајлове, а затим штампа њихове бројеве.

\subsubsection{llobdstat}
llobdstat програм приказује OST статистику за задати OST и временски интервал.
\begin{verbatim}
llobdstat ost_name [interval]
 \end{verbatim}

\subsubsection{llstat}
Кориснички програм за приказивање статистике је \textit{llstat}.
\begin{verbatim}
llstat [-c] [-g] [-i interval] stats_file
 \end{verbatim}
Опције овог програма су:

\begin{itemize}
\item \textbf{\textit{-c }}
Брише фајл у коме се чувају резултати статистике.
\item \textbf{\textit{-i} }
Подешава временски интервал.
\item \textbf{\textit{-g} }
Подешава графички приказ статистике.
\item \textbf{\textit{stats\_file }}
Подешава путању до фајла статистике или кратку референцу MDS или OST.
\end{itemize}



