\section{Kључне карактеристике}

\begin{itemize}
\item  \textbf{Скалабилност} - \textit{Lustre} перформансе зависе од броја клијентских чворова, складишта и пропусног опсега. Тренутно највећа инсталација   \textit{Lustre} система покренута у продукцији ради са 26000 клијента, на кластерима који имају између 10000-20000 клијената. Неколико   \textit{Lustre} система има капацитет од 1 PB  или више, омогућавајући складиште за 2 милијарде фајлова.

\item  \textbf{Перформансе} -   \textit{Lustre}  y продукцији има проток од око 100 GB/s. У тест окружењима перформансе су око 130 GB/s и 13,000 creates/s. 
%МИЛОШ: Објаснити ову новоуведену јединицу.
  \textit{Lustre}  клијент има проток око 2 GB/s и OSS проток од 2.5 GB/s (максимално). Такође, постоје подаци да је покренут на 240 GB/s на \textit{Spider} фајл систему у \textit{Oak Ridge National Laboratories}.

\item \textbf{\textit{POSIX} сагласност} - Потпуни \textit{POSIX} стандарди су  испуњени на   \textit{Lustre}  клијентима. Код кластера, \textit{POSIX} сагласност значи да су све операције једноставне и да клијенти увек приступају свежим подацима.

\item \textbf{Висока доступност} -   \textit{Lustre}  нуди дељено складиште OSS (за OST) и дељено складисте MDS (за MDT).

\item  \textbf{Сигурност} - У   \textit{Lustre} систему, TCP конекција се одвија само преко привилегованих портова. Припадност групама се одређује на серверу.  POSIX листе за контролу приступа (\textit{Access control lists - \gls{ACL}}) су такође подржане.

\item  \textbf{Бесплатан} -   \textit{Lustre}  је лиценциран под  GNU GPL.


\item  \textbf{Интероперативност} - \textit{Lustre} је покренут на више врста процесорских архитектура и на више верзија   \textit{Lustre} система истовремено. 

\item \textbf{Листе за контролу приступа} - Тренутно,   \textit{Lustre} безбедносни модел дозвољава UNIX фајл систем побољшан са \textit{POSIX ACLs (Access control list)}. Додатне функције укључују \textit{root squash} и конекцију само са привилегованих потрова.

\item  \textbf{Квоте} - Корисничке и групне квоте су доступне.

\item \textbf{OSS додатак} - Капацитет фајл система и проток кластера може бити повећан додавањем новог OSS са OST, и то без икаквих прекида услуге.

\item  \textbf{Контролисано дељење фајлова} - Подразумевани \textit{stripe count} и \textit{stripe size} може бити контролисан на неколико начина. Фајл систем добија стандардна подешавања приликом инсталације. Такође, директоријумима може бити додат атрибут који ће означавати начин дељења. Велики број библиотека и програма омогућавају једноставно контролисање дељења индивидуалних фајлова у   \textit{Lustre} системима.

\item  \textbf{Тренутни снимак - \textit{Snapshot}} -   \textit{Lustre}  фајл сервери користе \textit{volumes} који се налазе на серверским чворовима. У пакету   \textit{Lustre} програмa  се налази и програм који омогућава креирање  снимака свих \textit{volumes}.

\item \textbf{Алати за прављење резервних копија} -   \textit{Lustre}  подржава 2 услужна алата: Један алат скенира фајл систем и проналази измењене фајлове у датом временском периоду. Овај алат прави списак путања до измењених фајлова, а затим се ти фајлови обрађују паралелно користећи други алат. Веома  користан алат је и измењена верзија GNU tar (\textit{gtar}) која прави резервне копије и врши повратак проширених атрибута (за дељење фајла и дозволама за приступ). 
\end{itemize}

