\section{Преносни процес покретања}

Мали, али веома користан додатак MPI-2 стандарда је стандардни метод за покретање MPI програма, који у ранијем стандарду није био специфициран. Најједноставнији пример овог метода је

\begin{verbatim}
mpiexec -n 16 myprog
\end{verbatim}

за покретање програма \textit{myprog} на  16 процеса. MPI спецификација стриктно не говори како се покреће MPI програм, већ утиче на писање самог програма. Од MPI програма се захтева покретање на широком спектру окружења, различитим оперативним системима, менаџерима процеса итд. Све ово доводи до чињенице да механизам за мулти-процесно покретање није могућ.

Међутим, корисници желе да програме са једне машине покрену на другој машини без икаквих додатних подешавања. Неколико савремених MPI имплементација користи \textit{mpirun}  за покретање MPI послова. Команда \textit{mpirun} се разликује од имплементације до имплементације и захтева различите аргументе. То доводи до конфузијем поготово када су различите MPI имплементације инсталиране на истој машини.
Да би прекинули све недоумице, MPI форум је одлучио да се позабави и овим проблемом у верзији стандарда MPI-2.
Она препоручује да је  \textit{mpiexec} једини програм за покретање  MPI апликација и да су аргументи овог програма тачно одређени и јединствени.
Команда

\begin{verbatim}
mpiexec -n 32 myprog
\end{verbatim}

стартује 32 MPI процеса, где је величина \texttt{MPI\_COMM\_WORLD} комуникатора 32. Назив \textit{mpiexec} је изабран да се избегну сукоби са различитим варијантама mpirun програма.

Поред \textit{-n <numprocs>} argumentа, mpiexec има и један мали број аргумената који су одређени MPI стандаром. У сваком случају формат за аргументе је -<назив> вредност. Неки од осталих аргумената су:
\begin{itemize}
\item \textit{soft}
\item \textit{host}
\item \textit{arch}
\item \textit{wdir}
\item \textit{path}
\item \textit{file}
\end{itemize}

\begin{Verbatim}
mpiexec -n 32 -soft 16 myprog
\end{Verbatim}
Значи да уколико се због ограничења распоређивања процеса програм не може покренути на 32 процеса, онда да се покрене на 16 процеса.
\begin{Verbatim}
mpiexec -n 4 -host denali -wdir /home/me/outfiles myprog
\end{Verbatim}
Значи покренути 4 процеса на машини под називом denali и притом поставити радни директоријум на \textit{/home/me/outfiles}.

\begin{Verbatim}
mpiexec -n 12 -soft 1:12 -arch sparc-solaris \
-path /home/me/sunprogs myprog
\end{Verbatim}

Значи да уколико се не може покренути програм на 12 процеса, покренути програм на било ком броју процеса од 1 до 12, на \textit{sparc-solaris} архитектури, с тим што се програм \textit{myprog} налази у директоријуму \textit{/home/me/sunprogs}. 

\begin{Verbatim}
mpiexec -file myfile
\end{Verbatim}
Значи да \textit{mpiexec} погледа \textit{myfile} за следеће инструкције. Формат фајла зависи од MPI имплементације.
